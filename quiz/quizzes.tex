% arara: xelatex
\documentclass[12pt]{article}

% Stochastic Processes, 2024-2025

% \usepackage{physics}

\usepackage{zxjatype}
\usepackage[ipa]{zxjafont}
% https://tex.stackexchange.com/questions/15516/how-to-write-japanese-with-latex

\usepackage{hyperref}
\hypersetup{
    colorlinks=true,
    linkcolor=blue,
    filecolor=magenta,      
    urlcolor=cyan,
    pdftitle={Overleaf Example},
    pdfpagemode=FullScreen,
    }

\usepackage{tikzducks}

\usepackage{tikz} % картинки в tikz
\usetikzlibrary{shapes, arrows, positioning}
\usepackage{microtype} % свешивание пунктуации

\usepackage{array} % для столбцов фиксированной ширины

\usepackage{indentfirst} % отступ в первом параграфе

\usepackage{sectsty} % для центрирования названий частей
\allsectionsfont{\centering}

\usepackage{amsmath, amsfonts, amssymb} % куча стандартных математических плюшек

\usepackage{comment}

\usepackage[top=2cm, left=1.2cm, right=1.2cm, bottom=2cm]{geometry} % размер текста на странице

\usepackage{lastpage} % чтобы узнать номер последней страницы

\usepackage{enumitem} % дополнительные плюшки для списков
%  например \begin{enumerate}[resume] позволяет продолжить нумерацию в новом списке
\usepackage{caption}

\usepackage{url} % to use \url{link to web}


\newcommand{\smallduck}{\begin{tikzpicture}[scale=0.3]
    \duck[
        cape=black,
        hat=black,
        mask=black
    ]
    \end{tikzpicture}}

\usepackage{fancyhdr} % весёлые колонтитулы
\pagestyle{fancy}
\lhead{Time Series and Stochastic Processes}
\chead{}
\rhead{Quizzes for samurai}
\lfoot{刃に強き者は礼にすぐる (Ha ni tsuyoki-sha wa rei ni suguru) \\ Кто искусен в обращении с клинком, искусен и в вежливости.}
\cfoot{}
\rfoot{}

\renewcommand{\headrulewidth}{0.4pt}
\renewcommand{\footrulewidth}{0.4pt}

\usepackage{tcolorbox} % рамочки!

\usepackage{todonotes} % для вставки в документ заметок о том, что осталось сделать
% \todo{Здесь надо коэффициенты исправить}
% \missingfigure{Здесь будет Последний день Помпеи}
% \listoftodos - печатает все поставленные \todo'шки


% более красивые таблицы
\usepackage{booktabs}
% заповеди из докупентации:
% 1. Не используйте вертикальные линни
% 2. Не используйте двойные линии
% 3. Единицы измерения - в шапку таблицы
% 4. Не сокращайте .1 вместо 0.1
% 5. Повторяющееся значение повторяйте, а не говорите "то же"


\setcounter{MaxMatrixCols}{20}
% by crazy default pmatrix supports only 10 cols :)


\usepackage{fontspec}
\usepackage{libertine}
\usepackage{polyglossia}

\setmainlanguage{russian}
\setotherlanguages{english}

% download "Linux Libertine" fonts:
% http://www.linuxlibertine.org/index.php?id=91&L=1
% \setmainfont{Linux Libertine O} % or Helvetica, Arial, Cambria
% why do we need \newfontfamily:
% http://tex.stackexchange.com/questions/91507/
% \newfontfamily{\cyrillicfonttt}{Linux Libertine O}

\AddEnumerateCounter{\asbuk}{\russian@alph}{щ} % для списков с русскими буквами
% \setlist[enumerate, 2]{label=\asbuk*),ref=\asbuk*}

%% эконометрические сокращения
\DeclareMathOperator{\Cov}{\mathbb{C}ov}
\DeclareMathOperator{\Corr}{\mathbb{C}orr}
\DeclareMathOperator{\Var}{\mathbb{V}ar}
\DeclareMathOperator{\col}{col}
\DeclareMathOperator{\row}{row}

\let\P\relax
\DeclareMathOperator{\P}{\mathbb{P}}

\DeclareMathOperator{\E}{\mathbb{E}}
% \DeclareMathOperator{\tr}{trace}
\DeclareMathOperator{\card}{card}

\DeclareMathOperator{\Convex}{Convex}
\DeclareMathOperator{\plim}{plim}

\newcommand{\cF}{\mathcal{F}}
\newcommand{\cH}{\mathcal{H}}



\newcommand{\cN}{\mathcal{N}}
\newcommand{\RR}{\mathbb{R}}
\newcommand{\NN}{\mathbb{N}}
\newcommand{\hb}{\hat{\beta}}


\usepackage{mathtools}

\DeclarePairedDelimiter{\norm}{\lVert}{\rVert}
\DeclarePairedDelimiter{\abs}{\lvert}{\rvert}
\DeclarePairedDelimiter{\scalp}{\langle}{\rangle}
\DeclarePairedDelimiter{\ceil}{\lceil}{\rceil}



\begin{document}

\section*{Quiz 2}

\newpage
\begin{enumerate}
\item I throw a fair dice indefinitely often. 
If it shows 1 then the sum in the hat is burned and the game stops. 
If it shows 2 then I immediately get 2 roubles, the sum in the hat is burned and the game stops. 
If it shows 3 then the sum in the hat is burned and the game continues.
If it shows 4 then I immediately get 4 roubles and the game continues. 
It it shows 5 then 5 rouble is added to the hat and the game continues.
If it shows 6 nothing happens and the game continues. 

What is my expected total payoff?

\newpage
\item I throw a fair dice indefinitely often. 
If it shows 1 then the sum in the hat is burned and the game stops. 
If it shows 2 then I immediately get 2 roubles, the sum in the hat is burned and the game stops. 
If it shows 3 then the sum in the hat is burned and the game continues.
If it shows 4 then I immediately get 4 roubles plus the sum in the hat and the game stops. 
It it shows 5 then 5 rouble is added to the hat and the game continues.
If it shows 6 nothing happens and the game continues. 

What is my expected total payoff?


\newpage
\item Winnie-the-Pooh starts at zero on the real line. 
Every minute he moves one step to the left with probability $0.4$, one step to the right with probability $0.1$ or skips. 
If he reaches points $(-2)$ or $(2)$ he stays there forever. 

What is the expected duration of the game?

\newpage
\item Winnie-the-Pooh starts at zero on the real line. 
Every minute he moves one step to the left with probability $0.4$, one step to the right with probability $0.1$ or skips. 
If he reaches points $(-2)$ or $(2)$ he stays there forever. 

What is the probability that he will eventually rest at $(2)$?

\newpage
\item Alice throws a fair coin until the sequence HНH appears.

What is the expected duration of the game?

\newpage
\item Alice throws a fair coin until the sequence HHT appears.

What is the expected duration of the game?



\end{enumerate}

\end{document}

